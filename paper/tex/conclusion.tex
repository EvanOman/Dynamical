\chapter{Conclusions and Future Work}

This paper has introduced several results pertaining to the behavior of the critical orbit of a very special subset of the  rational maps of the plane. In the introduction we proposed a study of the larger two-dimensional, two parameter family $f_{c,\beta}$. While the results of this paper are not trivial, they largely pertain only to the real ``spine'' of this system where we held $\beta = .001$ and varied the complex parameter $c$, restricting $c$ to be real. As such, there are many open questions that remain:

\begin{itemize}
	\item This paper focused exclusively on the behavior of the critical orbit. While there are good reasons for doing so (see Schwarzian discussion in Section 3.1), a full description of the dynamics would require a discussion of the long term orbits of all points, not just the critical point. Thus the most natural future study would be to consider other orbits than the critical orbit. 

	\item In addition to the parameter accumulations described thus far, there seems to be additional accumulations for the real function. Looking at images like Figure 3.9 but with higher iterates, there seems to be a new accumulations of the $z_n$ and $p_n$ parameter values near every $h_n$ parameter value (where $f_{c}$ admits a critical homoclinic orbit). This is readily apparent by surveying the graphical iteration diagrams in Figures 3.10 and 3.11: every time there is a homclinic orbit, parameter values arbitrarily close to this value yield arbitrarily high order periodic and prezero orbits.

	\item We arbitrarily fixed the perturbation term $\beta $ at $.001$. We know that for $\beta < 0$, the asymptote is directed toward $-\infty$, and consequently there are no critical points (making our technique irrelevant: there are no critical orbits to consider). However it would be interesting to see which ranges of $\beta$ the described behavior holds and what other behaviors may arise as $\beta$ varies.


	\item In order to introduce the continuity arguments in Chapter 4,  we considered our map under a two-point compactification of the reals. When we plot the resulting function, it has a similar shape to a $4^{th}$ degree polynomial such as $x^4 - x^2$. While the maps are definitely very different, it would be interesting to see what structure they share.

	\item It would be interesting to see if there are any parallels to the $h_n$, $p_n$, and $z_n$ accumulations in the complex plane. Looking at Figures \ref{par} and \ref{zpar}, there seems to be patterns of escape regions in the real$\times$imaginary space which  connect the explained escape regions along the real axis. Based on the complexity of the one-dimensional map, these two-dimensional structures would likely be very interesting to explore.

\end{itemize}

 While this study focused on just the real part of a special family within this problem space, we were still able to uncover infinite levels of nontrivial behavior. It is this complexity, in addition to the new questions outlined above, which tells us 1) that there is a great amount of work to be done in order to understand all maps of the plane and 2) that these maps, even simplified special cases, can exhibit complex behavior which is worth studying.