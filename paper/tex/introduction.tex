\chapter{Introduction}

	A broad spectrum of Discrete Dynamical Systems research has been dedicated to understanding rational maps of the plane such that $f: \R^2 \ra \R^2$. The goal of any such study is to develop a description of the ``full dynamics'' of the chosen family of maps as stated below:

	\underline{\textbf{Full Dynamics of a Family of Maps}}

	{\itshape The full dynamics of some family of maps $f_{\alpha_1, \alpha_2, \ldots, \alpha_i}: S \ra S$ is the characterization of the long term behavior of every point $s \in S$ under $f$ as the parameters $\alpha_i$ vary.}

	One special family of maps of $\R^2$, defined in complex coordinates, is the quadratic family defined as follows:
	\begin{equation}
		z_n \mapsto Q_c (z_n) = z_n^2 + c = z_{n+1}
	\end{equation}
	
	where $z \in \C$. %This is more generally treated as a map of $\R^2$ when it is separated into its real and complex components. 
	 This family has been been studied extensively in both the one and two dimensional settings \cite{Dev1}. We will explore a singular perturbation of this well known family by adding an inverse square conjugate term with perturbation parameter $\beta$ which yields the system:

	\begin{equation}
	z_n \mapsto f_{c, \beta} (z_n) = z_{n}^2 + c + \frac{\beta}{\overline{z}_{n}^2} = z_{n+1}
	\end{equation}
	% \[\]

	Robert Devaney and others have studied very similar maps, in particular Devaney et al. consider the map $z^2 + c + \frac{\lambda}{z^2}$ in \cite{sim}. The family listed in Eq. 1.2 differs by adding a complex conjugate term, making the perturbation not only singular but also nonholomorphic (since it is well know that any map containing $\overline{z}$ cannot be holomorphic).   The goal of this study is to compare the well known dynamics of the standard quadratic family with the dynamics of our perturbed family in order to illuminate how the two families differ. We will begin with a summary of some results, definitions, and theorems from the field of Dynamical Systems. Next we present some one-dimensional results showing that the singularity introduces an infinity of significant parameter values, some corresponding to superattracting periodic orbits and others to parameter values for which the critical point maps to $\infty$ within a finite number of iterates. %infinite new windows of escape and accumulations of critical periodic orbits. 
	Finally, we conclude by briefly discussing how the one-dimensional results relate to the larger two-dimensional system.

\newpage
