\documentclass[12pt]{report}

\usepackage[margin=.5in]{geometry}
\usepackage{import, amsmath,amssymb}

\import{./}{headerStuffs.tex}

\newenvironment{myproof}[1][\proofname]{%
  \begin{proof}[#1]$ $\par\nobreak\ignorespaces
}{%
  \end{proof}
}

\newcommand{\Fcb}{F_{c, \beta}}


%\numberwithin{proposition}{chapter}
%\numberwithin{definition}{chapter}
%\numberwithin{theorem}{chapter}

\begin{document}

\begin{center}
	Paper 
\end{center}

\begin{definition}{\textbf{: Homoclinic Point (Devaney)}}
	Let $f (p) = p$ and $f' (p) > 1$. A point $q$ is called homoclinic to $p$ if $q \in W^u_{loc} (p)$ and there exists $n> 0$ such that $f^n (q) = p$. The point $q$ is heteroclinic if $q \in W^u_{loc} (p)$ and there exists $n > 0$ such that $f^n (q)$ lies on a different periodic orbit
\end{definition}

\begin{definition}
	A homoclinic orbit is called nondegenerate if $f' (x)\neq 0$ for all points $x$ on the orbit. Otherwise, the orbit is degenerate.
\end{definition}

\begin{theorem}
Suppose $q$ lies along a nondegenerate homoclinic orbit to a fixed point $p$. Then  for each neighborhood $U$ of $p$, there is an integer $n>0$ such that $f^n$ has a hyperbolic invariant subset in $U$ on which $f^n$ is topologically conjugate to the shift automorphism.
\end{theorem}

\begin{theorem}
Suppose $f$ admits a nondegenerate homoclinic point to $p$. Then, in every neighborhood of $p$, there are infinitely many distinct periodic point.
\end{theorem}

Quick conventions:
\begin{itemize}
	\item $C$: the right critical point of $\Fcb$ (the orbit of the left critical point, call it $C_L$, varies from $C$ by only one iterate)
	\item $F_r$: the right fixed point of $\Fcb$ (if it exists)
	\item $\Fcb^{-i} (x)$ represents the $i^{th}$ preimage of $x$.
\end{itemize}

This paper will explore some of the dynamics of the system $Fcb (x) = x^2 + c + \frac{\beta}{x^2}$.

\begin{myproof}[Some iterate entering ``trap door'' is the only cause of critical orbit escape]
Self evident?

Preimage of $R$ is the top part of the graph, which only includes the left and right branch of the singularity and the high enough parts of $L$. If I could prove that $C$ cannot get into that part of $L$ then I am set.
\end{myproof}

\begin{myproof}[``Trap Door'' escapes are framed by fre-fixed points at $F_r$]
	Depending on the sign of $\ds\frac{\partial \Fcb^n (C)}{\partial c}$, the point $\Fcb^n (C)$ will either march up and then down the singularity, or down then up. Either way, then end and start of these intervals is marked by the right fixed point $F_r$. Thus $\Fcb^n (C)$ must be prefixed at $F_r$ for some value $\delta$ to either side of the point where $\Fcb^n (C) = 0$
\end{myproof}

\begin{myproof}[Infinite Pre-Zero Points Following a Pre-Fixed Point]
Consider the orbit of 0 under the inverse map $F_{c, \beta}\inv$ where $\beta= .001$ and $c$ is the  parameter value some small $\varepsilon$ less than the point where the critical orbit coding is $Cr\bar{F_r}$ (experimentally this occurs at $c\approx -.0924$). Since $Fp_r$ is repelling in forward time (clearly the slope there is greater than 1), it must be attracting in backward time. Thus $\ds\lim_{n \ra \infty} F_{c,\beta}^{-n} (0) = F_r$. Then the orbit of $0$ under the inverse map would be the sequence of monotonically increasing points $\{\Fcb\inv (0), \Fcb^{-2} (0), \Fcb^{-2} (0) \ldots\}$ which limits to $F_r$. In forward time, this sequence of points is precisely those points which are pre-0 such that the $i^{th}$ iterate of $\Fcb^{-i} (0)$ is exactly 0.

Then we see that as $c$ decreases, $\Fcb^2 (C)$ decreases and moves away from $Fp_r$, going down the curve until we hit our first period 2.

\begin{itemize}
\item TODO: argument that it does so smoothly
\item TODO: argument that even though the location of our pre-zero points change for each c value, $\Fcb^2$ must still go through each pre-zero point
\item Would an argument saying that $\Fcb\inv$ and $\Fcb^2$ are continuous with respect to $c$ and that $\Fcb^2$ changes faster than $\Fcb\inv$ suffice?
\end{itemize}


Therefore from $c = -.0924$ to $c = -.1218$, $\Fcb^2 (C)$ takes on $\Fcb^{-i} (0)$ for all $i \geq 2$, meaning that the critical orbit has an infinite number of distinct escape windows on this interval.
\end{myproof}

\begin{center}
FROM DEVANEY'S BOOK, ABOUT LOGISTIC MAP
\end{center}


\begin{proposition}
	Let $F_{\mu} (x) = \mu (1-x)$.
	\alphen{
	\item $\ds \frac{d}{d\mu} \left (\frac{1}{2}\right) < 0$ if $\mu > \frac{8}{3}$
	\item $\ds\left (F^2_{\mu}\right)'' \left (\frac{1}{2}\right) = \mu^2 (\mu - 2)$.
	}
\end{proposition}

\begin{proof}
	PROOF
\end{proof}

\begin{proposition}
	\begin{enumerate}
	\item For each $j \geq 0$, $F^{j+2}_{\mu}$ has a unique critical point $c_j (\mu)$ in $I_j$, and $F^{j+2}$ has a minimum at this point. Also $F^{j+2}_{\mu} (c_j (\mu))=F^2_{\mu}\left (\frac{1}{2}\right)$.

	\item $F^{j+2}_{\mu} (x_j) = F^{j+2}_{\mu} (\hat{x_j}) = F^2_{\mu} (x_0)$
	\end{enumerate}
\end{proposition}

\begin{proof}
	For part 1, we note that on $I_j$, we have $F^{j+2}_{\mu} = F^2_{\mu}\circ F^j_{\mu}$. Hence 
	\[
		\left (F^{j+2}_{\mu}\right)' (x) = \left (F^2_{\mu}\right)'\left (F^j_{\mu} (x)\right)\cdot \left (F^j_{\mu}\right)' (x)
	\]
	Since $F^j_{\mu}: I_j \ra I_0$ is a diffeomorphism, $ (F_{\mu}^j)' (x) > 0$ for $x \in I_j$. Thus the only critical points occur where $ (F^2_{\mu})'$ vanishes, namely at $F^j_{\mu} = \frac{1}{2}$. The result now follows immediately. Part 2 follows from the fact that $F_{\mu} (x_0) = F_{\mu} (\hat{x_0})$ by symmetry.
\end{proof}



\end{document}